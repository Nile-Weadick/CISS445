\input{thispreamble.tex}

\renewcommand\AUTHOR{nweadick1@cougars.ccis.edu} % CHANGE TO YOURS

\begin{document}
\topmattertwo

If the question is not meaningful or the OCAML expression is invalid, write ERROR.

%------------------------------------------------------------------------------
\nextq
Write a function \texttt{prodint} such that \texttt{(prodint x y)} return
the product of integers \verb!x! and \verb!y!.
Do NOT use syntactic sugar.
This must be a complete ocaml expression that can run in the ocaml shell (so
don't forget the two semicolons).
\\
\ANSWER
\begin{answercode}
let prodint = fun x -> fun y -> x * y;;
\end{answercode}

%------------------------------------------------------------------------------
\nextq
Refer to the previous question. 
What is the type of \verb!prodint!?
\\
\ANSWER
\begin{answercode}
int -> int -> int
\end{answercode}

%------------------------------------------------------------------------------
\nextq
Refer to the previous question. 
What is the type of \verb!(prodint 2)!?
\\
\ANSWER
\begin{answercode}
int -> int
\end{answercode}

%------------------------------------------------------------------------------
\nextq
Refer to the previous question. 
What is the type of \verb!(prodint 2 42)!?
\\
\ANSWER
\begin{answercode}
int
\end{answercode}

%------------------------------------------------------------------------------
\nextq
What is the type of \verb!fun a -> fun b -> fun x -> a *. x +. b!?
\\
\ANSWER
\begin{answercode}
float -> float -> float -> float
\end{answercode}

%------------------------------------------------------------------------------
\nextq
What is the type of \verb!(fun c -> fun x -> if c then x + 1 else x - 1) true!?
\\
\ANSWER
\begin{answercode}
int -> int -> bool -> int -> int
\end{answercode}

%------------------------------------------------------------------------------
\newpage
\input{instructions.tex}
\end{document}


