\input{thispreamble.tex}

\renewcommand\AUTHOR{nweadick1@cougars.ccis.edu} % CHANGE TO YOURS

\begin{document}
\topmattertwo

%------------------------------------------------------------------------------
\nextq
The first phase of compilation is called scanning or 
\answerbox{Lexcial} analysis.

%------------------------------------------------------------------------------
\nextq
In first phase of compilation, the input is a \answerbox{character} stream.

%------------------------------------------------------------------------------
\nextq
In first phase of compilation, the output is a \answerbox{token} stream.

%------------------------------------------------------------------------------
\nextq
The second phase of compilation is called parsing or \answerbox{syntax} analysis.

%------------------------------------------------------------------------------
\nextq
In the second phase of compilation, the output from the first phase is used to produce a \answerbox{parse tree}.

%------------------------------------------------------------------------------
\nextq
All phases of a compilation rely on a \answerbox{symbol table} that keeps
track of all the identifiers in the program and what the
compiler knows about them.

%------------------------------------------------------------------------------
\newpage
\input{instructions.tex}
\end{document}


