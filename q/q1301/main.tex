\input{thispreamble.tex}

\renewcommand\AUTHOR{nweadick1@cougars.ccis.edu} % CHANGE TO YOURS

\begin{document}
\topmattertwo


%------------------------------------------------------------------------------
\nextq
Using tail recursion,
write a recursive function
\verb!last_value! such that
\verb!(last_value list)! computes the list containing the last value
of \verb!list!, or if \verb!list! is empty it computes \verb![]!.
For instance \verb!(last_value [1;3;5])! is \verb![5]!
and \verb!(last_value [])! is \verb![]!
\\
\ANSWER
\begin{answercode}
let rec last = fun list -> match list with [] -> [] 
| x::xs -> if(xs == []) then [x] else last (xs);;
\end{answercode}
\begin{answercode}
let last_value = fun list -> last(list);;
\end{answercode}

%------------------------------------------------------------------------------
\nextq
Using tail recursion,
write a recursive function
\verb!last_pair! such that
\verb!(last_pair list0 list1)! computes the list containing the
tuple of the last value of \verb!list0! and the last value of \verb!list1!;
if \verb!list0! or \verb!list1! is empty, the empty list is computed.
For instance \verb!(last_pair [1;3;5] [2;4;6])! is \verb![(5, 6)]!
and \verb!(last_pair [] [2;4;6])! is \verb![]!
\\
\ANSWER
\begin{answercode}
let rec last = fun list1 -> fun list2 -> match list1,list2 with [],[] -> [] 
| [],list2 -> [] 
| list1,[] -> [] 
| x::xs,y::ys -> if(xs == [] and ys == []) then x::y::[] else last (xs) (ys);;
\end{answercode}

\begin{answercode}
let last_pair = fun list1 -> fun list2 -> last (list1) (list2);;
\end{answercode}

%------------------------------------------------------------------------------
\newpage
\input{instructions.tex}
\end{document}
