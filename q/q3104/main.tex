\input{thispreamble.tex}

\renewcommand\AUTHOR{nweadick1@cougars.ccis.edu} % CHANGE TO YOURS

\begin{document}
\topmattertwo


%------------------------------------------------------------------------------
\nextq
\tf.
Design a DFA $M$ that accepts strings made up of $0$'s and $1$'s
that contains exactly one substring with a strictly positive even number of $0$'s
(of course \lq\lq strictly positive even number" means 2 or 4 or 6 or ...).
Here are some strings in $L(M)$:
\[
00, 0000, 100111, 1100, 111001111
\]
Here are some strings not in $L(M)$:
\[
\epsilon, 010, 000, 100000, 11110111, 1001001, 1100001110000
\]
Here's an example of now you should specify your DFA below:
\begin{Verbatim}[frame=single,fontsize=\small]
S = ["0", "1"]
Q = ["q0", "q1", "q2", "q3"]
start = "q0"
F = ["q1"]
transitions = {("q0","0"): "q1",
               ("q0","1"): "q2",
               ("q1","0"): "q0”,
               ("q1","1"): "q1",
               ("q2","0"): "q2",
               ("q2","1"): "q3",
               ("q3","0"): "q0",
               ("q3","0"): "q0",
              }
\end{Verbatim}
In this case, the set of symbols is $S$
which is made up of \verb!"0"! and \verb!"1"!.
The states are in $Q$ which in this case is made up of
\verb!"q0"!, ..., \verb!"q3"!.
The start state is \verb!"q0"!.
The accept states are in $F$ – in this case where's only one accept states
\verb!"q1"!.
The transitions are in \verb!transitions!.
For instance at state \verb!"q0"!,
if the character to be processed is \verb!"0"!,
then the DFA goes into state \verb!"q1"!.
(See the first entry for transition.)
If at state \verb!"q0"!, the character to be processes is \verb!"1"!,
then the DFA goes into state \verb!"q2"!. Etc.
The above is (of course) not the answer.
It's provided just to show you how to describe the DFA to me.
Make sure you follow the format exactly. 

\ANSWER
\begin{Verbatim}[frame=single,fontsize=\small]
S = ["0", "1"]
Q = ["q0", "q1"]
start = "q0"
F = ["q1"]
transitions = {("q0","00"): "q0",
                ("q0","1"): "q1",
                ("q1","0"): "q0”,
                ("q1","1"): "q1",
                }
\end{Verbatim}

(This DFA is easy enough for you to design it as a DFA.
But if you can't get the design right away, then describe the
regex, then the NFA, and then convert to DFA.)

%------------------------------------------------------------------------------
\newpage
\input{instructions.tex}
\end{document}
