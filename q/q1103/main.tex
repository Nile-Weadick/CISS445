\input{thispreamble.tex}

\renewcommand\AUTHOR{nweadick1@cougars.ccis.edu} % CHANGE TO YOURS

\begin{document}
\topmattertwo

%------------------------------------------------------------------------------
\nextq
Declare a function
\verb!sq! that computes the square of the float that is passed in.
You must write it using OCAML's lambda notation,
i.e. without syntactic sugar or you'll get 0.
(If I don't see \lq\lq \verb!fun!'' or \verb!->! then it's wrong ...
you get zero ... nada ... zippo ...zilch). Here's a sanity check:
\begin{console}
# sq 2.0 ;;
- : float = 4.
\end{console}
\\
\ANSWER
\begin{answercode}
let sq = fun x -> x *. x;
\end{answercode}

%------------------------------------------------------------------------------
\nextq
Without using syntactic sugar,
declare a function \verb!sumsq! that computes the sum of squares of two float
values.
For instance since $2.0^2 + 3.0^2$ is $13.0$:
\begin{console}
# sumsq 2.0 3.0;;
- : float = 13.0
\end{console}
You must write it using OCAML's lambda notation, i.e., without syntactic sugar.
You must use the function \verb!sq! above or you'll get 0.
\\
\ANSWER
\begin{answercode}
let sumsq = fun a -> fun b -> (a *. a) +. (b *. b);;
\end{answercode}


%------------------------------------------------------------------------------
\nextq
What is the type of the following expression:
\begin{console}
fun x -> if x < 0 then let a = 1 in x + a else let b = 2 in x - b
\end{console}
\\
\ANSWER
\begin{answercode}
int -> int
\end{answercode}

%------------------------------------------------------------------------------
\newpage
\input{instructions.tex}
\end{document}

